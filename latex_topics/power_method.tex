\documentclass[12pt]{article}
\usepackage{amsmath, tcolorbox}
\bibliographystyle{plain} 
\begin{document}

Let me outline the main points and the overall structure of the power methods described in \cite{borm2012numerical}.
Of course, I will add some points of myself to make some concepts more clear.

After establishing $A^{k}x$ converges to the eigenvector of $A$ corresponding to the largest eigenvalue, 
we need some stopping criterion, which is the main concern in numerical analysis because we cannot set max\textunderscore eiter simply. 

Since $x^{(m)}$ cannot be eigenvector identically, there is no real number $a$ satisfying $Ax^{(m)} = ax^{(m)}$. 
But we can expect the number $a$ which makes
$$
\|A x^{(m)}-ax\| 
$$
minimizes is a good approximation of the largest eigenvalue. Now we elaborate on how to solve this $a$.

\paragraph{Rayleigh Quotient.} If one see $Av = av$ as an overdetermined equation system, technique about \textbf{least square} can be applied
to get the solution $a$ which is called Rayleigh quotient.

\begin{equation}
    \label{RayleighQ}
    \Lambda_{A}(v) := \frac{v^{*}Av}{v^{*}v}
\end{equation}

Geometrically, the solution of least square is just to find a scalar $a$ such that $Av-av$ is orthogonal to $v$.

\begin{tcolorbox}[colback=gray!10, colframe=black, title=Shaded Box]
    One can also deduce the same result \eqref{RayleighQ} algebraically by solving the following optimization problem
    \begin{equation}
        \label{optimization1}
        \operatorname{min}_{x} \|Av-a v\|_{2}^{2},
    \end{equation}
    then eq \eqref{optimization1} (take real case as example) boils down to finding the critical point of
    $$
    f(a)=v^{T}v a^{2}- v^{T}(A^{T}+A)v a+v^{T}A^{T}Av
    $$
    If $v^{T}A^{T}v$ is a real number, then $v^{T}A^{T}v = v^{T}Av$, one can use the derivative of $f(a)$ to determine the critical point and the result matches the Least Square. 
    When we consider the complex number field, the expansion of $f(x)=\operatorname{min}_{x} \|Av-a v\|_{2}^{2}$ is subtle, saying
    $$
    f(a)\neq v^{*}v a^{2}- v^{*}(A^{*}+A)v a+v^{*}A^{*}Av
    $$
    especially when $a$ is complex.
    \end{tcolorbox}




The theoretical support is given by Theorem 4.6 in \cite{borm2012numerical}.

Combined with $\|x^{(m)}-v\|\approx 0$ and $|\lambda^{(m)} -\lambda|\approx 0$ where $\lambda, v\in\operatorname{eig}(A)$, we can see that
$$
\|Ax^{(m)}-\lambda^{(m)}x^{(m)}\| \to 0
$$
where each components can be computed explicitly in the process of iteration. So one option for stopping criterion is to set $\epsilon$ and stop the iteration when
$$
\|Ax^{(m)}-\lambda^{(m)x^{(m)}}\| < \epsilon
$$


\bibliography{D:/numerical-analysis/reference.bib} % The name of your .bib file without the .bib extension
\end{document}